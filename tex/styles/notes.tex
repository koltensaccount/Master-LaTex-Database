% ====================================================
% NOTES STYLE: Layout & Typography for the Textbook Notes
% ----------------------------------------------------
% Loads XeLaTeX-specific tooling, margin figure helpers,
% and decorative chapter/section elements.
% ====================================================

% --- NOTES: Page geometry & margin configuration ---
\usepackage[
  paperwidth=8.5in,
  paperheight=11in,
  twoside=false,
  left=2.5in,
  right=0.75in,
  top=0.25in,
  bottom=1in,
  headheight=26pt,
  includeheadfoot,
  marginparwidth=1.6in,
  marginparsep=15pt
]{geometry}

\usepackage{titlesec}
\usepackage{marginnote}
\usepackage{caption}
\usepackage{lipsum}
\usepackage{tocloft}
\usepackage{tocloft}

\makeatletter
\@ifundefined{ifMathNotesEmbedded}{\newif\ifMathNotesEmbedded}{}
\MathNotesEmbeddedfalse
\makeatother

% --- NOTES: Font configuration (XeLaTeX) ---
\usepackage{fontspec}
\newcommand{\NotesFontMono}{\ttfamily}
\IfFontExistsTF{JetBrains Mono}{%
  \newfontfamily\mathnotesmainfont{JetBrains Mono}
  \renewcommand{\NotesFontMono}{\mathnotesmainfont}
}{%
  \typeout{[notes] Font 'JetBrains Mono' not found; using default monospaced font.}%
}
\IfFontExistsTF{Helvetica}{%
  \newfontfamily\mathnotessansfont{Helvetica}
}{%
  \typeout{[notes] Font 'Helvetica' not found; using 'TeX Gyre Heros'.}%
  \newfontfamily\mathnotessansfont{TeX Gyre Heros}
}
\IfFontExistsTF{Times New Roman}{%
  \newfontfamily\mathnotesseriffont{Times New Roman}
}{%
  \typeout{[notes] Font 'Times New Roman' not found; using 'TeX Gyre Termes'.}%
  \newfontfamily\mathnotesseriffont{TeX Gyre Termes}
}

\newcommand{\NotesFontBody}{\mathnotesseriffont}
\newcommand{\NotesFontHeader}{\mathnotessansfont\bfseries}
\newcommand{\NotesFontTitle}{\mathnotessansfont\bfseries\LARGE}
\newcommand{\NotesFontCaption}{\mathnotesseriffont\small}
\newcommand{\NotesFontEmph}{\mathnotesseriffont\itshape}

\AtBeginDocument{%
  \NotesFontBody
  \reversemarginpar
}

\makeatletter
\@ifundefined{cftbeforechapskip}{}{%
  \setlength{\cftbeforechapskip}{0.75em}
  \renewcommand{\cftchapfont}{\NotesFontHeader}
  \renewcommand{\cftchappagefont}{\NotesFontHeader}
  \renewcommand{\cftchapleader}{\hfill}
  \renewcommand{\cftchapdotsep}{\cftnodots}
}
\@ifundefined{cftbeforesecskip}{}{%
  \setlength{\cftbeforesecskip}{0pt}
}
\@ifundefined{cftsecfont}{}{%
  \renewcommand{\cftsecfont}{\NotesFontBody}
}
\@ifundefined{cftsecpagefont}{}{%
  \renewcommand{\cftsecpagefont}{\NotesFontBody}
}
\newcommand{\NotesTableOfContents}{%
  \begingroup
    \let\contentsname\relax
    \@starttoc{toc}%
  \endgroup
}
\protected\def\NotesChapterTOCEntry#1#2{%
  \if@mainmatter
    \addcontentsline{toc}{chapter}{Chapter~#1\ \textemdash\ #2}%
  \else
    \addcontentsline{toc}{chapter}{Appendix~#1\ \textemdash\ #2}%
  \fi
}
\makeatother

% --- NOTES: Header & footer styling ---
\usepackage{fancyhdr}
\pagestyle{fancy}
\fancyhf{}
\fancyhead{}%
\fancyhead[L]{%
  \ifodd\value{page}
    \NotesFontHeader\MakeUppercase{\leftmark}%
  \else
    \thepage
  \fi
}
\fancyhead[R]{%
  \ifodd\value{page}
    \thepage
  \else
    \NotesFontHeader\MakeUppercase{\leftmark}%
  \fi
}
\renewcommand{\headrulewidth}{0.4pt}
\setlength{\headheight}{26pt}
\fancypagestyle{plain}{\pagestyle{fancy}}

% --- NOTES: Equation and custom counter alignment with chapters ---
\numberwithin{equation}{chapter}
\newcounter{textexample}[chapter]
\renewcommand{\thetextexample}{\thechapter.\arabic{textexample}}
\newcounter{notestheorem}[chapter]
\renewcommand{\thenotestheorem}{\thechapter.\arabic{notestheorem}}
\newcounter{notesdefinition}[chapter]
\renewcommand{\thenotesdefinition}{\thechapter.\arabic{notesdefinition}}
\newcounter{marginfigure}[chapter]
\renewcommand{\themarginfigure}{\thechapter.\arabic{marginfigure}}
\crefname{notestheorem}{Theorem}{Theorems}
\Crefname{notestheorem}{Theorem}{Theorems}
\crefname{notesdefinition}{Definition}{Definitions}
\Crefname{notesdefinition}{Definition}{Definitions}

% --- NOTES: Example block with chapter-aware numbering ---
\newcommand{\solutiontag}{%
  \tikz[baseline=(s.base)] \node[fill=Accent,draw=none,
    text=Primary,font=\NotesFontHeader,rounded corners=2pt,
    inner xsep=6pt,inner ysep=2pt] (s) {Solution};%
}

\newenvironment{textexample}[1]{%
  \refstepcounter{textexample}%
  \Needspace*{5\baselineskip}%
  \vspace{0.6em}\noindent
  \makebox[0pt][l]{\hspace*{-1.2in}%
    \begin{tikzpicture}[baseline=(title.base)]
      \def\BW{1in}\def\BH{0.72cm}%
      \node[inner sep=0pt,minimum width=\BW,minimum height=\BH,
            anchor=base west,fill=Highlight,rounded corners=2pt] (box) at (0,0) {};%
      \node[font=\NotesFontHeader,text=white,anchor=center]
        at (box.center){Example~\thetextexample};%
      \coordinate (ruleStart) at (box.north west);%
      \coordinate (ruleEnd) at ($(ruleStart)+(\textwidth+1.2in,0)$);%
      \draw[RuleGray,line width=1pt](ruleStart)--(ruleEnd);%
      \node[font=\NotesFontHeader\large,text=Primary,
            anchor=base west,right=0.2in of box.east] (title) {#1};%
    \end{tikzpicture}%
  }%
  \par\vspace{0.8em}\noindent
}{\vspace{0.8em}}

% --- NOTES: Decorative theorem/definition boxes ---
\tcbset{
  mathnotesbox/.style={
    colback=Accent,
    colframe=Accent,
    fonttitle=\NotesFontHeader\color{Primary},
    boxrule=0pt,
    arc=2pt,
    left=10pt,right=10pt,top=6pt,bottom=6pt,
    enhanced,
  }
}

\newenvironment{theoremBox}[1]{%
  \refstepcounter{notestheorem}%
  \begin{tcolorbox}[mathnotesbox,
    title={THEOREM~\thenotestheorem\hspace{0.5em}\normalfont\itshape#1}]%
}{\end{tcolorbox}}

\newenvironment{definitionBox}[1]{%
  \refstepcounter{notesdefinition}%
  \begin{tcolorbox}[mathnotesbox,
    title={DEFINITION~\thenotesdefinition\hspace{0.5em}\normalfont\itshape#1}]%
}{\end{tcolorbox}}

% --- Intake-specific grayscale blocks (lectures/discussions) ---
\newcounter{intakedefinition}
\newcounter{intaketheorem}
\newcounter{intakeexample}
\newcounter{intakesection}
\crefname{intakedefinition}{Definition}{Definitions}
\Crefname{intakedefinition}{Definition}{Definitions}
\crefname{intaketheorem}{Theorem}{Theorems}
\Crefname{intaketheorem}{Theorem}{Theorems}
\crefname{intakeexample}{Example}{Examples}
\Crefname{intakeexample}{Example}{Examples}
\crefname{intakesection}{Section}{Sections}
\Crefname{intakesection}{Section}{Sections}

\makeatletter
\newcommand{\mathnotes@resetintakecounters}{%
  \setcounter{intakedefinition}{0}%
  \setcounter{intaketheorem}{0}%
  \setcounter{intakeexample}{0}%
  \setcounter{intakesection}{0}%
}
\makeatother

\newcommand{\MathNotesIntakeSubtitle}[1]{%
  \if\relax\detokenize{#1}\relax
    \relax
  \else
    \hspace{0.5em}\normalfont\itshape#1%
  \fi
}

\newcommand{\MathNotesIntakeExampleTitle}[1]{%
  \if\relax\detokenize{#1}\relax
    \strut
  \else
    #1%
  \fi
}

\tcbset{
  intakebox/.style={
    colback=black!4,
    colframe=black!40,
    fonttitle=\NotesFontHeader\color{black},
    boxrule=0pt,
    arc=2pt,
    left=10pt,right=10pt,top=6pt,bottom=6pt,
    enhanced,
  }
}

\NewDocumentEnvironment{IntakeDefinition}{O{}}{%
  \refstepcounter{intakedefinition}%
  \begin{tcolorbox}[intakebox,
    title={DEFINITION~\theintakedefinition\MathNotesIntakeSubtitle{#1}}]%
    \NotesFontBody
}{\end{tcolorbox}}

\NewDocumentEnvironment{IntakeTheorem}{O{}}{%
  \refstepcounter{intaketheorem}%
  \begin{tcolorbox}[intakebox,
    title={THEOREM~\theintaketheorem\MathNotesIntakeSubtitle{#1}}]%
    \NotesFontBody
}{\end{tcolorbox}}

\NewDocumentEnvironment{IntakeExample}{O{}}{%
  \refstepcounter{intakeexample}%
  \Needspace*{5\baselineskip}%
  \vspace{0.6em}\noindent
  \makebox[0pt][l]{\hspace*{-1.2in}%
    \begin{tikzpicture}[baseline=(title.base)]
      \def\BW{1in}\def\BH{0.72cm}%
      \node[inner sep=0pt,minimum width=\BW,minimum height=\BH,
            anchor=base west,fill=black!45,rounded corners=2pt] (box) at (0,0) {};%
      \node[font=\NotesFontHeader,text=white,anchor=center]
        at (box.center){Example~\theintakeexample};%
      \coordinate (ruleStart) at (box.north west);%
      \coordinate (ruleEnd) at ($(ruleStart)+(\textwidth+1.2in,0)$);%
      \draw[black!35,line width=1pt](ruleStart)--(ruleEnd);%
      \node[font=\NotesFontHeader\large,text=black,
            anchor=base west,right=0.2in of box.east] (title) {\MathNotesIntakeExampleTitle{#1}};%
    \end{tikzpicture}%
  }%
  \par\vspace{0.8em}\noindent
  \NotesFontBody
}{\vspace{0.8em}}

\NewDocumentEnvironment{LectureDefinition}{O{}}{\begin{IntakeDefinition}[#1]}{\end{IntakeDefinition}}
\NewDocumentEnvironment{LectureTheorem}{O{}}{\begin{IntakeTheorem}[#1]}{\end{IntakeTheorem}}
\NewDocumentEnvironment{LectureExample}{O{}}{\begin{IntakeExample}[#1]}{\end{IntakeExample}}
\NewDocumentEnvironment{DiscussionDefinition}{O{}}{\begin{IntakeDefinition}[#1]}{\end{IntakeDefinition}}
\NewDocumentEnvironment{DiscussionTheorem}{O{}}{\begin{IntakeTheorem}[#1]}{\end{IntakeTheorem}}
\NewDocumentEnvironment{DiscussionExample}{O{}}{\begin{IntakeExample}[#1]}{\end{IntakeExample}}

% --- NOTES: Margin figure helpers ---
\newcommand{\MarginFigureAutoContent}[2]{%
  \begin{minipage}{\marginparwidth}
    \centering
    #1\\[2pt]
    {\NotesFontCaption\textbf{\color{Primary}FIGURE~\themarginfigure.} #2}%
  \end{minipage}
}
\newcommand{\MarginFigureAuto}[3][0pt]{%
  \refstepcounter{marginfigure}%
  \vspace{#1}%
  \marginnote{\MarginFigureAutoContent{#2}{#3}}%
  \label{fig:\themarginfigure}%
}

% --- NOTES: Chapter/section heading macros ---
\setcounter{chapter}{0}

\newcommand{\NotesSlugSet}[2]{\MathNotesSlugSet{#1}{#2}}

\makeatletter
\newif\ifmathnotes@intakeopen
\mathnotes@intakeopenfalse

\newcommand{\mathnotes@closeintake}{%
  \ifmathnotes@intakeopen
    \end{itemize}%
    \mathnotes@intakeopenfalse
  \fi
}

\newcommand{\mathnotes@startintake}{%
  \begin{itemize}[leftmargin=*,label=\textbullet,topsep=2pt,itemsep=2pt]
  \mathnotes@intakeopentrue
}

\newcommand{\mathnotes@ensureintakeopen}{%
  \ifmathnotes@intakeopen
  \else
    \mathnotes@startintake
  \fi
}

\NewDocumentCommand{\MathNotesIntakePoint}{O{} m}{%
  \mathnotes@ensureintakeopen
  \item \MathIfBlankTF{#1}{#2}{\textbf{#1:}~#2}%
}

\NewDocumentCommand{\LecturePoint}{O{} m}{\MathNotesIntakePoint[#1]{#2}}
\NewDocumentCommand{\DiscussionPoint}{O{} m}{\MathNotesIntakePoint[#1]{#2}}

\NewDocumentCommand{\IntakeSection}{O{} m}{%
  \mathnotes@closeintake
  \refstepcounter{intakesection}%
  \Needspace*{4\baselineskip}%
  \vspace{0.4em}\noindent
  {\NotesFontHeader\large #2}\par
  \vspace{0em}\noindent
  {\color{black!35}\rule{\textwidth}{0.6pt}}\par
  \vspace{0.3em}\noindent
  \phantomsection
  \if\relax\detokenize{#1}\relax\else
    \label{#1}%
  \fi
}

\NewDocumentCommand{\LectureSection}{O{} m}{\IntakeSection[#1]{#2}}
\NewDocumentCommand{\DiscussionSection}{O{} m}{\IntakeSection[#1]{#2}}

\newcommand{\closenotes}{\mathnotes@closeintake}
\newcommand{\opennotes}{\mathnotes@closeintake\mathnotes@startintake}

\NewDocumentCommand{\newlecture}{m m}{%
  \mathnotes@closeintake
  \NotesSlugSet{\mathnotes@titleslug}{#1}%
  \def\mathnotes@lectureheading{#2\ \textemdash\ #1}%
  \ifMathNotesEmbedded
    \section*{\mathnotes@lectureheading}%
    \addcontentsline{toc}{section}{Lecture\ #2\ ---\ #1}%
  \else
    \section{\mathnotes@lectureheading}%
  \fi
  \markboth{\mathnotes@lectureheading}{\mathnotes@lectureheading}%
  \phantomsection
  \label{lec:#2-\mathnotes@titleslug}%
  \mathnotes@resetintakecounters
}

\NewDocumentCommand{\newdiscussion}{m m m}{%
  \mathnotes@closeintake
  \NotesSlugSet{\mathnotes@packslug}{#1}%
  \NotesSlugSet{\mathnotes@sheetslug}{#2}%
  \def\mathnotes@discussionheading{Discussion\ #1\ ---\ #2\ (#3)}%
  \ifMathNotesEmbedded
    \section*{\mathnotes@discussionheading}%
    \addcontentsline{toc}{section}{Discussion\ #1\ ---\ #2\ (#3)}%
  \else
    \section{\mathnotes@discussionheading}%
  \fi
  \markboth{\mathnotes@discussionheading}{\mathnotes@discussionheading}%
  \phantomsection
  \label{disc:\mathnotes@packslug-\mathnotes@sheetslug-#3}%
  \mathnotes@resetintakecounters
}

\RenewDocumentCommand{\newhwprob}{m m O{}}{%
  \mathnotes@closeintake
  \begin{hwproblem}[#1]{#2}[#3]
}

\RenewDocumentCommand{\newdiscprob}{m m O{}}{%
  \mathnotes@closeintake
  \begin{discproblem}[#1]{#2}[#3]
}
\makeatother

\newcommand{\ChapterHeading}[2]{%
  \ifnum\value{chapter}>0
    \clearpage
  \fi
  \refstepcounter{chapter}%
  \setcounter{equation}{0}%
  \setcounter{textexample}{0}%
  \setcounter{notestheorem}{0}%
  \setcounter{notesdefinition}{0}%
  \setcounter{marginfigure}{0}%
  \vspace*{1em}%
  \begin{tikzpicture}[remember picture,overlay]
    \fill[Accent] (current page.west|-{0,0}) rectangle ++(\paperwidth,1.1cm);
  \end{tikzpicture}%
  \vspace{0.4cm}%
  \noindent
  \begin{minipage}{\textwidth}
    \raggedright
    \begin{tikzpicture}[baseline=(chapnum.base)]
      \node[fill=black,text=white,font=\NotesFontTitle,
            inner xsep=10pt,inner ysep=6pt,anchor=base west] (chapnum) {CHAPTER~\thechapter};
      \node[font=\NotesFontTitle,text=Primary,
            anchor=base west,right=0.8em of chapnum.base east,
            align=left,text width=\dimexpr\textwidth-4cm\relax] (title) {#2};
    \end{tikzpicture}
  \end{minipage}
  \markboth{#2}{#2}%
  \phantomsection
  \NotesChapterTOCEntry{\thechapter}{#2}%
  \vspace{0.4em}\par\noindent
}

\newcommand{\ContentsHeading}[2][]{%
  \def\contentsheadingdisplay{#2}%
  \def\contentsheadingmark{#2}%
  \if\relax\detokenize{#1}\relax\else
    \def\contentsheadingmark{#1}%
  \fi
  \vspace*{1em}%
  \begin{tikzpicture}[remember picture,overlay]
    \fill[Accent] (current page.west|-{0,0}) rectangle ++(\paperwidth,1.1cm);
  \end{tikzpicture}%
  \vspace{0.4cm}%
  \noindent
  \begin{minipage}{\textwidth}
    \raggedright
    \begin{tikzpicture}[baseline=(chapnum.base)]
      \node[fill=black,text=white,font=\NotesFontTitle,
            inner xsep=10pt,inner ysep=6pt,anchor=base west] (chapnum) {CONTENTS};
      \node[font=\NotesFontTitle,text=Primary,
            anchor=base west,right=0.8em of chapnum.base east,
            align=left,text width=\dimexpr\textwidth-4cm\relax] (title) {\contentsheadingdisplay};
    \end{tikzpicture}
  \end{minipage}
  \markboth{\contentsheadingmark}{\contentsheadingmark}%
  \vspace{2em}\par\noindent
}

\newcommand{\StartMainMatter}{%
  \clearpage
  \mainmatter
}

\DeclareRobustCommand{\InlineMacro}[1]{%
  \begingroup
  \ttfamily\small\sloppy
  \Urlmuskip=0mu plus 1mu\relax
  \nolinkurl{\detokenize{#1}}%
  \endgroup
}
\DeclareRobustCommand{\InlineCode}[1]{\InlineMacro{#1}}
\newenvironment{CodeBlock}{%
  \begin{adjustwidth}{0pt}{0pt}%
  \ttfamily\small\obeyspaces\obeylines\sloppy
  \setlength{\parindent}{0pt}%
  \rightskip=0pt plus 2em\relax
}{%
  \end{adjustwidth}
}

\newcommand{\SectionBar}[2]{%
  \Needspace*{4\baselineskip}%
  \vspace{0.4em}%
  \noindent
  \begin{tikzpicture}[baseline=(num.base)]
    \node[fill=black,text=white,font=\NotesFontHeader\large,
          inner xsep=8pt,inner ysep=4pt,anchor=base west] (num) {#1};
    \draw[RuleGray,line width=1pt]
      (num.south west)--($(num.south west)+(\textwidth,0)$);
    \node[font=\NotesFontHeader\large,text=Primary,
          anchor=base west,right=0.7em of num.base east] (txt) {#2};
  \end{tikzpicture}
  \vspace{0.4em}\par\noindent
  \phantomsection
  \addcontentsline{toc}{section}{\protect\numberline{#1}#2}%
}
