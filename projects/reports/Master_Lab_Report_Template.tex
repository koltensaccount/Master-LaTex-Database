% !TEX root = ./Master_Lab_Report_Template.tex
% !TEX program = latexmk
\makeatletter
\def\input@path{{./}{../}{../../}{../../../}{../../../../}}
\makeatother
\def\LabReportDocOptions{11pt,letterpaper}
% ====================================================
% Standalone Lab Report Document Wrapper
% ----------------------------------------------------
% Loads the shared lab-report style from tex/styles/
% and exposes begin/end helpers so individual reports
% only focus on content. Embedded builds toggle
% \ifLabReportEmbedded to skip the standalone branch.
% ====================================================

\makeatletter
\providecommand{\LabReportLocateTexPrefix}{%
  \def\LabReportTexPrefix{}%
  \@for\labreport@base:=,../,../../,../../../,../../../../,../../../../../\do{%
    \ifx\LabReportTexPrefix\@empty
      \IfFileExists{\labreport@base tex/core/colors.tex}{%
        \xdef\LabReportTexPrefix{\labreport@base tex/}%
      }{}%
    \fi
  }%
  \ifx\LabReportTexPrefix\@empty
    \PackageError{document-report}{Cannot locate tex/ directory}{Ensure this wrapper is included from within the repository tree or define \string\LabReportTexPrefix\space manually.}%
  \fi
}
\makeatother

\providecommand{\LabReportDocClass}{article}
\providecommand{\LabReportDocOptions}{11pt,letterpaper}
\providecommand{\LabReportDocSetup}{}

\providecommand{\LabReportDocumentBegin}{}
\providecommand{\LabReportDocumentEnd}{}
\providecommand{\LabReportStandaloneEnd}{}

\makeatletter
\@ifundefined{ifLabReportEmbedded}{%
  \newif\ifLabReportEmbedded
  \LabReportEmbeddedfalse
}{}
\makeatother

\ifLabReportEmbedded
  \renewcommand{\LabReportDocumentBegin}{}
  \renewcommand{\LabReportDocumentEnd}{}
  \renewcommand{\LabReportStandaloneEnd}{}
  \LabReportDocSetup
\else
  \documentclass[\LabReportDocOptions]{\LabReportDocClass}
  \LabReportLocateTexPrefix
  % ====================================================
% CORE: Color Palette Configuration
% ----------------------------------------------------
% - Define shared colors once for the entire project.
% - Override defaults by redefining \MathColor<Name>Model /
%   \MathColor<Name>Spec *before* this file is loaded.
% - Override on a per-document basis after loading via
%   \MathColorOverride{Name}{model}{spec}.
% ====================================================

% --- Default color specifications (override-friendly) ---
\usepackage[dvipsnames]{xcolor}

\providecommand{\MathColorPrimaryModel}{rgb}
\providecommand{\MathColorPrimarySpec}{0.18,0.24,0.50}

\providecommand{\MathColorAccentModel}{rgb}
\providecommand{\MathColorAccentSpec}{0.96,0.97,0.99}

\providecommand{\MathColorRuleGrayModel}{gray}
\providecommand{\MathColorRuleGraySpec}{0.80}

\providecommand{\MathColorHighlightModel}{rgb}
\providecommand{\MathColorHighlightSpec}{0.80,0.40,0.20}

\providecommand{\MathColorLinkModel}{rgb}
\providecommand{\MathColorLinkSpec}{0.18,0.24,0.50}

% --- Internal helper: define a named color from stored specs ---
\newcommand{\MathColorDefine}[1]{%
  \definecolor{#1}{\csname MathColor#1Model\endcsname}{\csname MathColor#1Spec\endcsname}%
}

\MathColorDefine{Primary}
\MathColorDefine{Accent}
\MathColorDefine{RuleGray}
\MathColorDefine{Highlight}
\MathColorDefine{Link}

% --- Public helper: override a single palette entry on demand ---
\newcommand{\MathColorOverride}[3]{%
  \expandafter\def\csname MathColor#1Model\endcsname{#2}%
  \expandafter\def\csname MathColor#1Spec\endcsname{#3}%
  \MathColorDefine{#1}%
}

% --- Hyperref defaults share the palette; adjust per document as desired ---
\AtBeginDocument{%
  \hypersetup{%
    colorlinks=true,
    linkcolor=Primary,
    urlcolor=Primary,
    citecolor=Primary,
    pdfborder={0 0 0}
  }%
}

  % ====================================================
% CORE: Shared Packages & Macros
% ----------------------------------------------------
% These resources are used by both the textbook-style notes
% and the printable homework documents.
% ====================================================

% --- CORE: Foundational packages ---
\usepackage{amsmath,amssymb,amsthm,amsfonts,amscd}
\usepackage{mathtools}
\usepackage{enumitem}
\usepackage{needspace}
\usepackage{xstring}
\usepackage{xparse}
\usepackage{float}
\usepackage{cancel}
\usepackage[superscript]{cite}
\usepackage{changepage}
\usepackage{soul}
\usepackage{tocloft}
\usepackage{hyperref}
\usepackage[capitalise,nameinlink]{cleveref}

\makeatletter
\g@addto@macro\UrlBreaks{%
  \do\/\do\\\do\_\do\-\do\{\do\}%
  \do\.\do:\do\=\do+\do\#\do\&\do\,\do\?%
}
\makeatother

\newcommand{\MathNotesSlugSet}[2]{%
  \begingroup
    \edef\MathNotesSlugTmp{\detokenize{#2}}%
    \lowercase{\edef\MathNotesSlugTmp{\MathNotesSlugTmp}}%
    \StrSubstitute{\MathNotesSlugTmp}{ }{-}[\MathNotesSlugTmp]%
    \StrSubstitute{\MathNotesSlugTmp}{~}{-}[\MathNotesSlugTmp]%
    \StrSubstitute{\MathNotesSlugTmp}{:}{-}[\MathNotesSlugTmp]%
    \StrSubstitute{\MathNotesSlugTmp}{/}{-}[\MathNotesSlugTmp]%
    \StrSubstitute{\MathNotesSlugTmp}{(}{}[\MathNotesSlugTmp]%
    \StrSubstitute{\MathNotesSlugTmp}{)}{}[\MathNotesSlugTmp]%
    \StrSubstitute{\MathNotesSlugTmp}{,}{}[\MathNotesSlugTmp]%
    \StrSubstitute{\MathNotesSlugTmp}{.}{}[\MathNotesSlugTmp]%
    \StrSubstitute{\MathNotesSlugTmp}{--}{-}[\MathNotesSlugTmp]%
    \StrSubstitute{\MathNotesSlugTmp}{--}{-}[\MathNotesSlugTmp]%
    \StrSubstitute{\MathNotesSlugTmp}{--}{-}[\MathNotesSlugTmp]%
    \xdef#1{\MathNotesSlugTmp}%
  \endgroup
}

% --- CORE: Graphics and plotting ---
\usepackage{graphicx}
\usepackage{tikz}
\usepackage{circuitikz}
\usetikzlibrary{calc,positioning,shapes.geometric}
\usepackage{pgfplots}
\pgfplotsset{compat=1.18}
\usepgfplotslibrary{groupplots}

% --- CORE: Box/enclosure support ---
\usepackage[most]{tcolorbox}

% --- CORE: Math shorthands ---
\newcommand{\vect}[1]{\mathbf{#1}}
\newcommand{\mat}[1]{\begin{bmatrix}#1\end{bmatrix}}
\newcommand{\R}{\mathbb{R}}
\newcommand{\rank}{\operatorname{rank}}
\newcommand{\nulls}{\operatorname{null}}
\newcommand{\Span}{\operatorname{span}}
\newcommand{\eig}{\operatorname{eig}}
\newcommand{\diff}{\mathrm{d}}

% --- CORE: Structural theorem environments ---
\theoremstyle{definition}
\newtheorem{definition}{Definition}[section]
\newtheorem{theorem}{Theorem}[section]
\newtheorem{example}{Example}[section]

% --- CORE: Homework problem + solution boxes ---
\newcounter{hwproblem}
\newcounter{discproblem}

\crefname{hwproblem}{Problem}{Problems}
\Crefname{hwproblem}{Problem}{Problems}
\crefname{discproblem}{Problem}{Problems}
\Crefname{discproblem}{Problem}{Problems}

\newcommand{\MathIfBlankTF}[3]{%
  \if\relax\detokenize{#1}\relax
    #2%
  \else
    #3%
  \fi
}

\makeatletter
\newcommand{\CurrentHomeworkSection}{}
\NewDocumentCommand{\HomeworkSection}{m}{%
  \section*{Section~#1}%
  \def\CurrentHomeworkSection{#1}%
}

\NewDocumentEnvironment{hwproblem}{O{} m O{}}{%
  \def\hwproblem@section{#1}%
  \def\hwproblem@number{#2}%
  \def\hwproblem@title{#3}%
  \Needspace{6\baselineskip}%
  \par\vspace{0.5em}%
  \begin{samepage}%
  \begin{adjustwidth}{2em}{0em}%
  \refstepcounter{hwproblem}%
  \edef\hwproblem@label{prob:hw-\hwproblem@section-\hwproblem@number}%
  \edef\@currentlabel{HW~\hwproblem@section.\hwproblem@number}%
  \MathIfBlankTF{#3}{%
    \protected@edef\@currentlabelname{Problem~HW~\hwproblem@section.\hwproblem@number}%
  }{%
    \protected@edef\@currentlabelname{Problem~HW~\hwproblem@section.\hwproblem@number~(#3)}%
  }%
  \phantomsection
  \label{\hwproblem@label}%
  \MathIfBlankTF{#3}{%
    \noindent\textbf{Problem\ \hwproblem@number}%
  }{%
    \noindent\textbf{Problem\ \hwproblem@number\ \textemdash\ #3}%
  }%
  \par\vspace{0.5em}%
  \setlength{\parindent}{0pt}%
}{%
  \end{adjustwidth}%
  \end{samepage}%
  \vspace{1em}%
}

\NewDocumentEnvironment{discproblem}{O{} m O{}}{%
  \def\discproblem@pack{#1}%
  \def\discproblem@id{#2}%
  \def\discproblem@topic{#3}%
  \Needspace{6\baselineskip}%
  \par\vspace{0.5em}%
  \begin{samepage}%
  \begin{adjustwidth}{2em}{0em}%
  \refstepcounter{discproblem}%
  \edef\discproblem@label{prob:disc-\discproblem@pack-\discproblem@id}%
  \edef\@currentlabel{Disc~\discproblem@pack.\discproblem@id}%
  \MathIfBlankTF{#3}{%
    \protected@edef\@currentlabelname{Problem~Disc~\discproblem@pack.\discproblem@id}%
  }{%
    \protected@edef\@currentlabelname{Problem~Disc~\discproblem@pack.\discproblem@id~(#3)}%
  }%
  \phantomsection
  \label{\discproblem@label}%
  \noindent\textbf{Disc\ \discproblem@pack\ \textbullet{}\ Problem\ \discproblem@id}%
  \MathIfBlankTF{#3}{}{%
    \ \textemdash\ #3%
  }%
  \par\vspace{0.5em}%
  \setlength{\parindent}{0pt}%
}{%
  \end{adjustwidth}%
  \end{samepage}%
  \vspace{1em}%
}

\newenvironment{solution}{%
  \par\vspace{0.25em}\noindent\textbf{Solution. }\setlength{\parindent}{0pt}\ignorespaces
}{\par\vspace{0.5em}}

\newcommand{\probref}[1]{\cref{#1}}
\newcommand{\Probref}[1]{\Cref{#1}}

\NewDocumentCommand{\newhwprob}{m m O{}}{\begin{hwproblem}[#1]{#2}[#3]}
\NewDocumentCommand{\newdiscprob}{m m O{}}{\begin{discproblem}[#1]{#2}[#3]}
\NewDocumentEnvironment{autohwproblem}{m O{}}{%
  \begin{hwproblem}[\CurrentHomeworkSection]{#1}[#2]
}{%
  \end{hwproblem}
}
\makeatother

\newenvironment{textbookproblem}[1][]{
  \Needspace{6\baselineskip}%
  \par\vspace{0.5em}%
  \begin{samepage}
  \begin{adjustwidth}{2em}{0em}%
  \noindent\textbf{#1}\par\vspace{0.5em}%
  \setlength{\parindent}{0pt}%
}{
  \end{adjustwidth}
  \end{samepage}
  \vspace{1em}%
}

\newtcolorbox{solutionbox}{
  enhanced,
  breakable,
  colback=white,
  colframe=black!15,
  borderline west={3pt}{0pt}{Highlight},
  sharp corners,
  boxrule=0pt,
  left=10pt, right=10pt, top=6pt, bottom=8pt,
  before skip=10pt, after skip=10pt,
  title={},
  fontupper=\normalsize,
  before upper={%
    \parindent=0pt\setlength{\emergencystretch}{2em}\setlength{\arraycolsep}{4pt}%
    \textcolor{Primary}{\textbf{Solution}}\par\medskip
  },
}

% --- CORE: Utility macros shared across styles ---
\newcommand{\keyword}[1]{\textbf{#1}}
\newcommand{\bluearrow}{\textcolor{Primary}{\large\(\blacktriangleright\)}\;}

  \usepackage{\LabReportTexPrefix styles/report/lab-report-template}
  \LabReportDocSetup
  \renewcommand{\LabReportDocumentBegin}{\begin{document}}
  \renewcommand{\LabReportDocumentEnd}{\end{document}}
  \renewcommand{\LabReportStandaloneEnd}{\LabReportDocumentEnd}
\fi


% Update these metadata values per report to keep headers and the cover page in sync.
\SetLabReportHeader{[Report Name or Number]}
\SetLabReportCoverTitle{Type Your Title Here:}
\SetLabReportCoverSubtitle{Concise and Informative Subtitle}
\SetLabReportPartners{[Lab Partners]}
\SetLabReportPreparedOn{\today}

\LabReportDocumentBegin

\MakeLabReportCoverPage
% Document body continues through Sections below.

% ==============================
% Abstract
% ==============================
\begin{center}
\section*{\centering Abstract}
\begin{adjustwidth}{1.5cm}{1.5cm}
\textit{[150–250 words summarizing purpose, method, quantitative results (with units), and conclusion. No citations.]}
\end{adjustwidth}
\end{center}

% ==============================
% Introduction
% ==============================
\Needspace{3\baselineskip}
\section*{Introduction}
[Provide scientific context, relevant theory, and objectives. Use equations like $\Delta T_f = iK_f m$. Cite key references.\cite{ref1}]

% ==============================
% Experimental
% ==============================
\Needspace{3\baselineskip}
\section*{Experimental}
[Summarize procedure in paragraph form, past tense. Include chemicals, instruments, and conditions. Enough detail for reproducibility.]

% ==============================
% Results
% ==============================
\Needspace{3\baselineskip}
\section*{Results}
[Present data clearly with tables and figures. Units in headers. Include one sample calculation. Summarize results concisely.]

% --- Example Table ---
\begin{table}[h]
\centering
\caption{Measured freezing points of solutions}
\begin{tabular}{lcc}
\hline
Solution & Concentration (mol/kg) & Freezing Point (°C) \\
\hline
NaCl     & 0.50                   & -1.85 \\
NaCl     & 1.00                   & -3.72 \\
Urea     & 1.00                   & -1.86 \\
\hline
\end{tabular}
\end{table}

% --- Example TikZ Diagram ---
\begin{figure}[h]
\centering
\begin{tikzpicture}[scale=1]
\draw[thick,->] (0,0) -- (5,0) node[right] {x-axis};
\draw[thick,->] (0,0) -- (0,4) node[above] {y-axis};
\draw[thick] (0,0) -- (4,3) node[midway, above] {Sample line};
\end{tikzpicture}
\caption{Simple diagram drawn with TikZ.}
\end{figure}

% --- Example PGFPlot (Styled) ---
\begin{figure}[h]
\centering
\begin{tikzpicture}
\begin{axis}[
    width=0.75\textwidth,
    xmin=0, xmax=1.9,
    ymin=0, ymax=6.2,
    xlabel={Molality (mol/kg)},
    ylabel={$\Delta T_f$ (°C)},
    tick align=outside,
    grid=both,
    grid style={line width=.3pt, draw=gray!25},
    major grid style={line width=.5pt, draw=gray!50},
    minor grid style={line width=.2pt, draw=gray!15},
    legend style={draw=black, fill=white, font=\small, fill opacity=0.8},
    legend cell align={left},
    legend pos=south east
]
\addplot[
    only marks,
    mark=*,
    mark size=3pt,
    draw=\LabReportSecondaryColor,
    fill=\LabReportSecondaryColor!60!white
] coordinates {
    (0.5,1.85) (1.0,3.72) (1.5,5.60)
};
\addlegendentry{Data}
\addplot[thick, domain=0:1.9, draw=\LabReportPrimaryColor] {3.53*x};
\addlegendentry{Best Linear Fit Line}
\end{axis}
\end{tikzpicture}
\caption{Freezing‐point depression versus molality with best‐fit line.}
\end{figure}

% ==============================
% Discussion
% ==============================
\Needspace{3\baselineskip}
\section*{Discussion}
[Interpret data. Compare with literature. Discuss sources of error. Connect back to theory and objectives.]

As shown in Figure~\ref{fig:titration-curve}, the pH rises sharply near the equivalence point, consistent with the expected titration behavior.

\begin{figure}[h]
\centering
\begin{tikzpicture}
\begin{axis}[
    width=0.75\textwidth,
    height=6cm,
    xmin=0, xmax=25,
    ymin=2, ymax=12,
    xlabel={Volume of Titrant (mL)},
    ylabel={pH},
    tick align=outside,
    grid=both,
    grid style={line width=.3pt, draw=gray!25},
    major grid style={line width=.5pt, draw=gray!50},
    minor grid style={line width=.2pt, draw=gray!15},
    grid style={line width=.3pt, draw=gray!25},
    major grid style={line width=.5pt, draw=gray!50},
    clip=false,
    legend style={at={(1.04,0.5)}, anchor=west, draw=black, fill=white, font=\small, fill opacity=0.9, align=left},
    legend cell align={left},
    samples=200,
    domain=0:25
]
% Sigmoidal titration curve with sharp transition near 12.5 mL
\addplot[thick, draw=\LabReportPrimaryColor] (
    x,
    {2 + 10/(1 + exp(-0.154*(x - 12.5)))}
);
\addlegendentry{Titration Curve}

% Equivalence point vertical dashed line
\draw[thick, dashed, color=\LabReportSecondaryColor] (axis cs:12.5,2) -- (axis cs:12.5,12);
\addlegendimage{thick, dashed, color=\LabReportSecondaryColor}
\addlegendentry{Equivalence Point}

% Buffer region vertical dashed lines and shaded area
\draw[thick, dashed, color=\LabReportPrimaryColor!50] (axis cs:2.5,2) -- (axis cs:2.5,12);
\draw[thick, dashed, color=\LabReportPrimaryColor!50] (axis cs:10,2) -- (axis cs:10,12);
\addlegendimage{thick, dashed, color=\LabReportPrimaryColor!50}
\addlegendentry{Buffer Region}
\addplot [ name path=bufferbottom, domain=2.5:10, draw=none, forget plot ] { 2 };
\addplot [ name path=buffertop, domain=2.5:10, draw=none, forget plot ] { 12 };
\addplot [ fill=\LabReportPrimaryColor, fill opacity=0.15, forget plot ] fill between [ of=bufferbottom and buffertop ];

% pKa horizontal line, marker, and label
% 1. Plot the horizontal line
\addplot [ thick, dashed, color=\LabReportAccentColor, domain=0:25 ] { 4.76 };
% 2. Add the legend entry for the line above
\addlegendentry { $ pK_a $ }

% 3. Plot the marker (dot) at half-equivalence
% We add 'forget plot' so it doesn't get its own legend entry
\addplot [ 
  only marks, 
  mark=*,
  mark size= 2.3pt ,
  draw= \LabReportAccentColor ,
  fill= \LabReportAccentColor,
  forget plot
] coordinates { (6.25,4.76) } ;

% Labels removed to avoid clutter
%\node[anchor=south west, color=\LabReportSecondaryColor, font=\small] at (axis cs:12.5,12) {Equivalence Point};
%\node[anchor=south west, color=\LabReportPrimaryColor!50, font=\small] at (axis cs:2.5,12) {Buffer Region};
%\node[anchor=south west, color=\LabReportPrimaryColor!50, font=\small] at (axis cs:10,12) {Buffer Region};
\node[anchor=north east, color=\LabReportAccentColor, font=\small] at (axis cs:25,4.76) {$pK_a$};
\end{axis}
\end{tikzpicture}
\caption{Example titration curve with key regions highlighted.}
\label{fig:titration-curve}
\end{figure}

% ==============================
% Conclusion
% ==============================
\Needspace{3\baselineskip}
\section*{Conclusion}
[Summarize findings concisely. Confirm whether objectives were achieved. No new data.]

% ==============================
% References
% ==============================
\newpage
\Needspace{3\baselineskip}
\begin{thebibliography}{9}
\bibitem{ref1} Author, A. B.; Author, C. D. \textit{Journal Name} \textbf{Year}, \textit{Volume}, page–page.  
\bibitem{ref2} Author, E. F. \textit{Book Title}; Publisher: Place, Year.  
\bibitem{ref3} Author, G. H. Title of Webpage. URL (accessed Sept 29, 2025).
\end{thebibliography}

\LabReportDocumentEnd
