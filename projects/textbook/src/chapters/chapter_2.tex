\ChapterHeading{2}{TeX Library Reference}
\label{ch:tex-library}

\SectionBar{2.1}{Module Overview}
\label{sec:tex-lib}

Everything under \InlineCode{tex/} is a reusable module. The directory is split
into semantic groups so a reader can map code to effects immediately:
\begin{description}[leftmargin=2.4em,labelwidth=2.2em]
  \item[\texttt{core/}] Foundational packages, typography helpers, and shared
    macros. These files define the vocabulary used by every project.
  \item[\texttt{styles/}] Visual identity for the textbook and homework formats.
    The geometry, headers, and display environments all live here.
  \item[\texttt{modules/}] Opt-in shims. Right now the only module toggles the
    homework style into ``embedded'' mode before streaming content into the book.
  \item[\texttt{system/}] Bootstrap helpers and the standalone wrappers that
    hide documentclass boilerplate for homework, intake notes, and lab reports.
\end{description}
Each sub-section below documents why every line in these modules exists and how
future changes should be staged.

\SectionBar{2.2}{\texttt{tex/core/colors.tex}: Palette Bootstrap}
\label{sec:core-colors}

The colour module keeps the palette declarative:
\begin{enumerate}[leftmargin=2em]
  \item \InlineMacro{\usepackage[dvipsnames]{xcolor}} loads the extended colour
        names once for the entire toolkit.
  \item Five \InlineMacro{\providecommand} calls define override-friendly storage
        slots for the semantic colours (\InlineCode{Primary}, \InlineCode{Accent},
        \InlineCode{RuleGray}, \InlineCode{Highlight}, and \InlineCode{Link}).
        Using \InlineMacro{\providecommand} instead of \InlineMacro{\newcommand}
        means downstream documents can reassign a colour before this file loads.
  \item \InlineMacro{\MathColorDefine} is a helper macro that converts the stored
        model/spec pair into a named colour via \InlineMacro{\definecolor}.
        Calling it for each palette entry keeps the public API tiny.
  \item \InlineMacro{\MathColorOverride} exposes a single entry point for runtime
        overrides. It updates both the stored model/spec pair \emph{and} calls
        \InlineMacro{\MathColorDefine} so later macros see the new value.
  \item The \InlineMacro{\AtBeginDocument} block syncs \InlineMacro{\hypersetup}
        with the palette so hyperlinks match the rest of the design.
\end{enumerate}
Every document that wants a different palette does so by redefining the stored
spec values before loading this file—no manual \InlineMacro{\definecolor}
clutter appears elsewhere.

\SectionBar{2.3}{\texttt{tex/core/base.tex}: Common Packages and Macros}
\label{sec:core-base}

This file centralises the dependencies shared by both projects.
\begin{description}[leftmargin=2.6em,labelwidth=2.4em]
  \item[\textbf{Math \& layout packages}] The opening \InlineMacro{\usepackage}
        block imports the AMS suite, \texttt{mathtools} (fixes alignment quirks),
        list utilities (\texttt{enumitem}), float helpers, \texttt{needspace} for
        widow/orphan control, and quality-of-life packages such as
        \texttt{hyperref}. Loading them once prevents mismatched versions between
        the book and homework PDFs.
  \item[\textbf{Graphics stack}] \texttt{graphicx}, \texttt{tikz}, and
        \texttt{pgfplots} are configured together so the compatibility level and
        libraries stay aligned. Any figure produced in one document will compile
        in the other.
  \item[\textbf{Utility macros}] The \InlineMacro{\vect}, \InlineMacro{\mat},
        \InlineMacro{\R}, and similar shorthands keep mathematical expressions
        readable. Each macro exists because it appears somewhere in the sample
        content; if you add a new convenience wrapper, document it in this
        section and inside the homework guide.
  \item[\textbf{Environment definitions}] \InlineMacro{\newtheorem} entries align
        numbering with the current section, while the custom
        \InlineMacro{\textbookproblem} and \InlineMacro{\solutionbox}
        environments format exercises and answers consistently across projects.
\end{description}
Because all maths-centric dependencies sit here, downstream documents only need
to manage content—they rarely touch this file unless a new global macro is
introduced.

\SectionBar{2.4}{\texttt{tex/styles/notes.tex}: Book Layout}
\label{sec:styles-notes}

The notes style translates the project into a textbook aesthetic:
\begin{enumerate}[leftmargin=2em]
  \item \textbf{Geometry block.} The \InlineMacro{\usepackage[...]{geometry}}
        call defines the wide left margin (for commentary), narrow top margin,
        and increased \InlineMacro{\marginparwidth}. Every numeric choice is the
        result of iterating on readability—changing it here updates the whole
        book.
  \item \textbf{Font selection.} A trio of \InlineMacro{\IfFontExistsTF} checks
        prefer JetBrains Mono, Helvetica, and Times New Roman while falling back
        to TeX Gyre alternatives. The macros \InlineMacro{\NotesFontBody},
        \InlineMacro{\NotesFontHeader}, etc. centralise font usage so the rest of
        the code speaks in roles, not typefaces.
  \item \textbf{Headers and footers.} The \InlineMacro{\pagestyle{fancy}} block
        creates mirrored headers that swap chapter titles and page numbers on odd
        and even pages. Setting \InlineMacro{\headheight} twice prevents common
        LaTeX warnings.
  \item \textbf{Counters.} \InlineMacro{\numberwithin} and the custom counters
        (\InlineCode{textexample}, \InlineCode{notestheorem}, etc.) ensure every
        displayed element tracks the chapter number automatically.
  \item \textbf{Environments.} The \InlineMacro{\begin{textexample}} definition
        draws the highlight banner while preserving line breaks; the theorem and
        definition boxes share a \texttt{tcolorbox} style so they always inherit
        the palette.
  \item \textbf{Headings.} \InlineMacro{\ChapterHeading}, \InlineMacro{\ContentsHeading},
        and \InlineMacro{\SectionBar} render the decorative chapter and section
        bars. They encapsulate TikZ drawing commands so authors never have to
        think about coordinates.
\end{enumerate}
Whenever you adjust typography or spacing, update the explanatory bullets above
and commit the rationale with the change. That practice keeps the guide in sync
with the code.

\SectionBar{2.5}{\texttt{tex/styles/homework.tex}: Assignment Layout}
\label{sec:styles-homework}

The homework style mirrors the book without duplicating logic:
\begin{enumerate}[leftmargin=2em]
  \item \textbf{Embedding toggle.} Lines 9--25 declare the boolean
        \InlineMacro{\ifMathHomeworkEmbedded}. Standalone PDFs leave it false,
        but the embedding module flips the flag before loading this file.
  \item \textbf{Palette adjustments.} \InlineMacro{\HomeworkApplyStandalonePalette}
        recolours accents when printing assignments on their own so pages remain
        legible on monochrome printers.
  \item \textbf{Geometry.} The \InlineMacro{\usepackage{geometry}} block is
        conditionally loaded only when producing standalone PDFs. Embedded mode
        inherits the book geometry to avoid conflicting header widths.
  \item \textbf{Header metadata.} \InlineMacro{\HomeworkSetup} stores course and
        assignment information; \InlineMacro{\HomeworkHeader} formats it within a
        \texttt{fancyhdr} header. The pagestyle is applied automatically at
        \InlineMacro{\AtBeginDocument}.
  \item \textbf{Print layout helper.} The optional print mode uses
        \InlineMacro{\HomeworkEnablePrintLayout} to reserve vertical space after
        each \InlineMacro{\solutionbox}, making handwritten answers neat.
  \item \textbf{Embedding helper.} \InlineMacro{\HomeworkIncludeInNotes} wraps
        every embedded assignment: it increments a counter for numbering,
        injects a table-of-contents entry, temporarily tweaks colours, and then
        restores the palette after the content finishes streaming in.
\end{enumerate}
The implementation looks long because each behaviour is isolated and commented.
This separation pays off when you only need to override one aspect—say, the
header metadata—without touching print layout logic.

\SectionBar{2.6}{\texttt{tex/modules/homework-embed.tex}: Embedding Shim}
\label{sec:module-homework}

This module contains just enough logic to reuse the homework style inside the
book:
\begin{enumerate}[leftmargin=2em]
  \item It ensures \InlineMacro{\ifMathHomeworkEmbedded} exists and then forces
        it to \texttt{true}. That flag prevents the homework geometry from
        loading and lets the book own the page layout.
  \item The module assumes \InlineMacro{\TexInput} is defined. When the textbook
        loads it, \InlineMacro{\TexInput} expands to the repository-aware import
        helper documented in Section~\ref{sec:top-level-layout}. If you compile
        the module in isolation (for regression tests), the fallback definition
        simply delegates to \InlineMacro{\input}.
  \item Finally, it loads \InlineCode{styles/homework}. No other side effects
        occur—the caller remains in charge of geometry, counters, and headers.
\end{enumerate}

Despite its size, this file is the connective tissue between the two deliverable
types. Without it, homework exercises would either break the textbook layout or
lose their consistent styling.

\SectionBar{2.7}{\texttt{tex/system/document-*.tex}: Standalone Wrappers}
\label{sec:system-wrappers}

The wrappers under \InlineCode{tex/system/} eliminate repetitive guards in the
project directories:
\begin{description}[leftmargin=2.6em,labelwidth=2.4em]
  \item[\textbf{document-homework.tex}] Declares
        \InlineCode{\HomeworkDocumentBegin/\HomeworkDocumentEnd},
        loads the homework style when compiled directly, and becomes a no-op
        when \InlineMacro{\ifMathHomeworkEmbedded} is true. Homework sources now
        contain only content and metadata via \InlineMacro{\HomeworkSetup}.
  \item[\textbf{document-notes.tex}] Handles the lecture/discussion intake
        workflow. Authors set a title/author/date, input the wrapper, and start
        capturing notes—the wrapper selects the article class, loads
        \InlineCode{tex/styles/notes.tex}, applies the standalone header, and
        calls \InlineMacro{\MathNotesDocumentEnd} for them.
  \item[\textbf{document-report.tex}] Performs the same role for lab reports,
        wiring the shared \InlineCode{lab-report-template.sty} into both
        standalone PDFs and appendix inclusions.
  \item[\textbf{bootstrap.tex}] Still exposes the repository-aware
        \InlineMacro{\TexInput} helper, but most day-to-day files no longer need
        to reimplement its detection logic—the wrappers import it once.
\end{description}

Adding a new deliverable now means ``write content + point at the wrapper'',
which keeps onboarding tight and enforces a single source of truth for every
document type.

\SectionBar{2.7}{Cross-Referencing Intake Material}
\label{sec:notes-usage-examples}

The snippet below mirrors the reference template: it combines a numbered figure,
a styled definition, and a theorem box. The surrounding prose demonstrates how a
curated chapter cites real intake material such as \Probref{prob:hw-3.4-23} from
Homework~5 and the discussion scratch work in \probref{prob:disc-W7-Q3}.

\begin{figure}[H]
  \centering
  \begin{tikzpicture}[scale=1.1]
    \draw[gray!40,->] (-3.5,0) -- (3.5,0) node[right] {$x$};
    \draw[gray!40,->] (0,-1.8) -- (0,1.8) node[above] {$y$};
    \draw[Primary,thick,smooth,samples=120,domain=-3.2:3.2]
      plot (\x,{0.9*sin(deg(\x))}) node[pos=0.92,above right] {$y=\sin x$};
    \draw[Accent,thick,dashed,samples=120,domain=-3.2:3.2]
      plot (\x,{0.6*cos(deg(\x))}) node[pos=0.18,below] {$y=\tfrac{3}{5}\cos x$};
  \end{tikzpicture}
  \caption{Sine and cosine responses used when discussing frequency-domain limits.}
  \label{fig:notes-main-plot}
\end{figure}

\MarginFigureAuto{%
  \begin{tikzpicture}[scale=0.9]
    \draw[gray!40,->] (-2.4,0) -- (2.4,0) node[right] {$x$};
    \draw[gray!40,->] (0,-1.3) -- (0,1.3);
    \draw[Primary,thick,samples=100,domain=-2.2:2.2]
      plot (\x,{0.7*sin(deg(\x))});
  \end{tikzpicture}
}{Margin sketch that mirrors the main plot for quick scanning.}

\begin{definitionBox}{Gain--Bandwidth Product}
\label{def:notes-gain-bandwidth}
The \emph{gain--bandwidth product} is the closed-loop gain of an amplifier
multiplied by the frequency at which that gain is measured. It captures how
fast a controller may respond before stability erodes.
\end{definitionBox}

\begin{theoremBox}{Single-Pole Closed-Loop Response}
\label{thm:notes-single-pole}
With low-frequency gain $G_0$ and dominant pole $\omega_p$, driving the network
captured in \cref{fig:notes-main-plot} produces the transfer function
\[
  G(j\omega) = \frac{G_0}{1 + j\omega/\omega_{\mathrm{p,eq}}}, \qquad
  \omega_{\mathrm{p,eq}} = \omega_p + \frac{1}{RC}.
\]
\end{theoremBox}

\begin{textexample}{Interpreting Frequency Sweeps}
\label{ex:notes-frequency-sweep}
\[
  \left|G(j\omega)\right| = \frac{G_0}{\sqrt{1 + (\omega/\omega_{\mathrm{p,eq}})^2}}
\]
\solutiontag\; Sample a handful of frequencies and compare the magnitudes against
\Probref{prob:hw-3.4-23}. The homework entry spells out the algebra behind the
matrix manipulations, while \probref{prob:disc-W7-Q3} retains the scratch work
that led to the approximation. Referencing Definition~\ref{def:notes-gain-bandwidth}
keeps the terminology inline with the discussion notes.
\end{textexample}

Citing the homework entry via \Probref{prob:hw-3.4-23} lets the chapter summarise
the takeaway without repeating every row operation. Likewise,
\probref{prob:disc-W7-Q3} preserves the spontaneous discussion notes that
eventually produced \cref{thm:notes-single-pole}. By keeping figures, theorem
boxes, and cross-references aligned with the intake appendices, you preserve the
original aesthetic while keeping the capture workflow fast.
