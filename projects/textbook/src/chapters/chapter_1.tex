\ChapterHeading{1}{Repository Layout and Build Entry Points }
\label{ch:system-overview}

\SectionBar{1.1}{Top-Level Directory Map}
\label{sec:top-level-layout}

The project root is intentionally shallow so a new contributor can discover the
build targets without digging through source first. Each entry below lists the
exact reason the directory exists.

\begin{description}[leftmargin=2.6em,labelwidth=2.4em]
  \item[\texttt{README.md}] Human-facing overview. Mirrors the structure in this
    guide so GitHub visitors immediately understand how to get started.
  \item[\texttt{.gitignore}] Keeps generated LaTeX artefacts out of version control.
    Every pattern is enumerated in Section~\ref{sec:output-hygiene}.
  \item[\texttt{docs/}] Markdown quick-start and architecture notes for readers
    who prefer plain text over the LaTeX guide. These files echo the content of
    Chapter~\ref{ch:system-overview} and Chapter~\ref{ch:authoring-workflow}.
  \item[\texttt{projects/}] All compilable deliverables.
    \begin{description}[leftmargin=2.8em,labelwidth=2.6em]
      \item[\texttt{textbook/}] The documentation-as-book project.
        \InlineCode{src/} hosts the LaTeX sources. Passing
        \InlineCode{-outdir=../output} keeps the book's artefacts in a sibling
        directory, mirroring the historic layout.
      \item[\texttt{homework/}] Content shared between the appendices and the
        standalone assignments. Each \InlineCode{homeworkXX.tex} file is
        self-contained but detects when it is embedded inside the textbook.
    \end{description}
  \item[\texttt{tex/}] The reusable TeX library, organised as modules:
    \InlineCode{core/} (packages + helpers), \InlineCode{styles/} (visual
    identity), and \InlineCode{modules/} (small shims such as homework embedding).
\end{description}

\SectionBar{1.2}{Textbook Build Flow}
\label{sec:textbook-pipeline}

The documentation book compiles through \texttt{projects/textbook/src/main.tex}.
That file performs the following steps every time XeLaTeX runs:
\begin{enumerate}[leftmargin=2em]
  \item \textbf{Resolve repository-relative paths.} Lines 11--35 compute the
        value of \InlineMacro{\TexRoot} and \InlineMacro{\HomeworkRoot} so the
        build works whether you call \InlineCode{latexmk main.tex} from inside
        \InlineCode{src/} or from the repository root.
  \item \textbf{Load the TeX library.} The helper macro \InlineMacro{\TexInput}
        wraps every include so the exact module path stays declarative:
        \InlineCode{core/colors}, \InlineCode{core/base},
        \InlineCode{styles/notes}, and finally the homework embedding shim from
        \InlineCode{modules/homework-embed}. Section~\ref{sec:tex-lib} gives the
        rationale for each module.
  \item \textbf{Render the documentation chapters.} The front matter styles and
        emits the live table of contents via
        \InlineCode{chapters/table-of-contents.tex} before inputting the curated
        chapters that form this tutorial.
  \item \textbf{Switch into appendix mode.} The build captures the default
        header width, swaps to homework geometry, emits the appendix chapter,
        streams in every embedded assignment, and then restores the book layout.
\end{enumerate}

Because every content include uses \InlineMacro{\TexInput} or
\InlineMacro{\HomeworkIncludeInNotes}, there is no hidden state—each line points
directly at its dependency. The only code outside the loop is the geometry swap
that protects the textbook layout from homework overrides.

\SectionBar{1.3}{Homework Projects}
\label{sec:homework-workflows}

Homework material now uses a single-source pattern: each file inside
\InlineCode{projects/homework/} begins with
\InlineCode{\textbackslash input\{../../tex/system/document-homework.tex\}}.
That wrapper loads the shared library, exposes
\InlineCode{\HomeworkDocumentBegin/\HomeworkDocumentEnd}, and
becomes a no-op when the textbook imports the same source via
\InlineCode{\HomeworkIncludeInNotes}. You edit every problem once and choose
whether to emit a standalone PDF or append it to the book simply by toggling
the entry point. Discussion, lecture, and lab-report capture reuse the same
philosophy via \InlineCode{tex/system/document-notes.tex} and
\InlineCode{tex/system/document-report.tex}.

To compile Homework~5 on its own:
\begin{CodeBlock}
\detokenize{cd projects/homework}
\detokenize{latexmk -xelatex -interaction=nonstopmode -halt-on-error -file-line-error homework05.tex}
\end{CodeBlock}
The wrapper resolves \InlineCode{\TexRoot} identically to the textbook, so
styling stays in sync across both targets and metadata helpers only live in one
place. Running latexmk from any project folder automatically mirrors the source
path by creating a sibling directory named after the `.tex` file (e.g.,
\InlineCode{projects/homework/homework05/homework05.*}), leaving the
\InlineCode{projects/} tree tidy.

\SectionBar{1.4}{Output Hygiene and Path Helpers}
\label{sec:output-hygiene}

Two small but important details keep the repository approachable:
\begin{enumerate}[leftmargin=2em]
  \item \textbf{Output isolation.} Standalone projects keep artefacts inside a
        folder named after the `.tex` file in the same directory (while the
        textbook build can optionally target \InlineCode{../output/}). Nothing
        under those folders is committed—contributors rebuild on demand instead
        of trusting stale binaries.
  \item \textbf{Path bootstrap.} The macros defined in
        \texttt{projects/textbook/src/main.tex} and
        \texttt{tex/system/document-*.tex} (homework, notes, reports)
        resolve the repository layout dynamically. They all expose the same
        helper:
\begin{CodeBlock}
\detokenize{\newcommand{\TexInput}[1]{\input{\TexRoot#1}}}
\end{CodeBlock}
        Any future document can reuse that pattern to opt into the shared TeX
        library without guessing relative paths. If the layout ever changes,
        update the detection logic in one place and the entire toolchain keeps
        working.
\end{enumerate}
