\ChapterHeading{3}{Authoring Workflow}
\label{ch:authoring-workflow}

\SectionBar{3.1}{Build Tooling and Commands}
\label{sec:build-tooling}

The toolkit relies on \texttt{xelatex} (for Unicode-friendly typesetting) and
optionally \texttt{latexmk} (for incremental builds). The commands below assume
you run them from the repository root; feel free to adapt them into shell
aliases or editor tasks.

\begin{description}[leftmargin=2.6em,labelwidth=2.4em]
  \item[Compile the textbook once]

        \begin{CodeBlock}
\detokenize{cd projects/textbook/src}
\detokenize{xelatex -interaction=nonstopmode -output-directory ../output main.tex}
        \end{CodeBlock}

        The explicit \InlineCode{-output-directory} flag mirrors the structure
        explained in Section~\ref{sec:output-hygiene}: every build artefact ends
        up alongside the source in \InlineCode{../output/}.

  \item[Continuous documentation build]

        \begin{CodeBlock}
\detokenize{cd projects/textbook/src}
\detokenize{latexmk -xelatex -interaction=nonstopmode -outdir=../output main.tex}
        \end{CodeBlock}

        \texttt{latexmk} watches the \InlineCode{src/} tree and only recompiles
        what changed. The flags mimic the single-run command so log parsing stays
        consistent across both approaches.

  \item[Compile a standalone homework]

        \begin{CodeBlock}
\detokenize{cd projects/homework}
\detokenize{latexmk -xelatex -interaction=nonstopmode -halt-on-error -file-line-error homework05.tex}
        \end{CodeBlock}

        Swap \InlineCode{hw05} for any other assignment file. latexmk creates a
        sibling directory named after the `.tex` file (e.g.,
        \InlineCode{projects/homework/homework05/homework05.*}) so the
        command above keeps artefacts next to the source without littering the
        root. Discussion, lecture, and report templates pick up the same
        behaviour automatically.
\end{description}

Re-run XeLaTeX (or let \texttt{latexmk} handle it) after changing geometry,
headers, or table-of-contents entries; a second pass updates cross-references
and hyperlink targets.

\SectionBar{3.2}{Adding New Content}
\label{sec:adding-content}

Both documentation and homework templates encourage copious inline explanation.
Follow the steps in this section so the structure remains predictable.

\subsection*{Adding a documentation chapter}
\begin{enumerate}[leftmargin=2em]
  \item Duplicate an existing file under
        \InlineCode{projects/textbook/src/chapters/} and rename it
        (for example, \InlineCode{chapter_4.tex}).
  \item Update the call to \InlineMacro{\ChapterHeading}, set unique labels with
        \InlineMacro{\label}, and immediately add a short paragraph explaining the
        goal of the chapter. Every code sample should mention which file it
        modifies so readers can trace the change.
  \item No manual table-of-contents edits are necessary—the wrapper file simply
        provides the styled heading and defers to \InlineMacro{\tableofcontents}.
  \item Add an \InlineCode{\input} line to
        \InlineCode{projects/textbook/src/main.tex}, placing it with the other
        chapter imports. The build script intentionally keeps the chapter list in
        one place so reviewers can validate the reading order quickly.
\end{enumerate}

\subsection*{Adding homework content}
\begin{enumerate}[leftmargin=2em]
  \item Create a new file in \InlineCode{projects/homework/}—the file should
        contain nothing but problems and solutions wrapped in the environments
        documented in Section~\ref{sec:core-base}. The guard at the top of each
        file is now the single line
        \InlineCode{\textbackslash input\{../../tex/system/document-homework.tex\}},
        so the wrapper decides whether to run the standalone preamble.
  \item Update the appendix list in
        \InlineCode{projects/textbook/src/main.tex}. Append another
        \InlineMacro{\HomeworkIncludeInNotes} entry with the display name and the
        relative path to the homework file. The order of these calls controls the
        numbering in the appendix table of contents.
\end{enumerate}

All homework copies—embedded, standalone, and print-mode—share the same source
file thanks to the document-homework wrapper. This design eliminates the drift
that typically happens when exercises are maintained in multiple locations.

\subsection*{Capturing lecture or discussion notes}
\begin{enumerate}[leftmargin=2em]
  \item Duplicate the thin templates under
        \InlineCode{projects/lecture/lecture-example.tex} or
        \InlineCode{projects/discussion/discussion-example.tex}. Each file simply
        sets metadata and inputs
        \InlineCode{../../tex/system/document-notes.tex}, so the dual-use guard
        lives in one place.
  \item Use \InlineMacro{\newlecture} (or \InlineMacro{\newdiscussion}) to start a
        section. The macros open a bullet list automatically; reach for
        \InlineMacro{\LecturePoint} or \InlineMacro{\DiscussionPoint} to append
        timestamped bullets without thinking about formatting.
  \item When you need inline scratch work, drop into the lightweight
        \InlineMacro{\begin\{LectureExample\}...\end\{LectureExample\}} or
        \InlineMacro{\begin\{DiscussionDrill\}...\end\{DiscussionDrill\}} helpers.
        These counters are local to the intake templates so they never conflict
        with the numbered theorem/example environments in the polished chapters.
  \item Grayscale versions of the chapter boxes—\InlineMacro{\begin\{LectureDefinition\}}, 
        \InlineMacro{\begin\{LectureTheorem\}}, and \InlineMacro{\begin\{LectureExample\}} 
        (with matching discussion aliases)—live in \InlineCode{tex/styles/notes.tex}.
        They never touch the main chapter counters, so you can stamp quick 
        definitions or theorems during class without polluting the curated numbering.
\end{enumerate}

\SectionBar{3.3}{Embedding Homework in the Documentation}
\label{sec:embedding-homework}

The appendix logic documented in Section~\ref{sec:styles-homework} and
Section~\ref{sec:module-homework} is deliberate. When you add a new assignment:
\begin{enumerate}[leftmargin=2em]
  \item Confirm the homework file compiles on its own (with the standalone
        preamble active). This sanity check ensures solution boxes, theorem
        counters, and colour overrides all behave as expected before the
        textbook pulls the file in.
  \item Use \InlineMacro{\HomeworkIncludeInNotes} with an optional display name if
        you need a shorter table-of-contents entry. The macro automatically:
        \begin{itemize}[leftmargin=1.5em]
          \item bumps the appendix counter,
          \item pushes a TOC entry,
          \item inserts a section bar that matches the rest of the book, and
          \item restores the standard palette when the homework finishes.
        \end{itemize}
  \item Keep spacing consistent. The macro temporarily redefines
        \InlineMacro{\newpage} and \InlineMacro{\clearpage} to avoid blank pages in
        the appendix. Do not reintroduce those commands inside the homework source
        unless a page break is absolutely necessary.
\end{enumerate}

Treat each assignment as documentation in its own right: annotate
non-obvious solution steps and reference any shared macros the problems rely on.

\SectionBar{3.4}{Per-Document Overrides}
\label{sec:document-overrides}

When the shared modules are almost—but not quite—what you need, apply local
overrides that respect the layering described in Chapter~\ref{ch:tex-library}.

\begin{itemize}[leftmargin=2em]
  \item \textbf{Palette experiments.} Redefine colour specs before calling
        \InlineMacro{\TexInput\{core/colors\}} (Section~\ref{sec:core-colors}).
        For single-document experiments, wrap the override in a group so the
        change does not leak into subsequent inputs.
  \item \textbf{Geometry tweaks.} Use the pattern from
        \InlineCode{projects/textbook/src/main.tex}: capture the current
        \InlineMacro{\headwidth}, enter a \InlineMacro{\newgeometry} block, and
        restore everything afterwards. This keeps embedded content from mutating
        the global layout.
  \item \textbf{Conditional content.} The homework style exposes the boolean
        \InlineMacro{\ifMathHomeworkEmbedded}. Use it to swap instructions or hide
        print-only banners when the same content is embedded in the book.
  \item \textbf{Custom macros.} Add small helpers to
        \InlineCode{tex/core/base.tex} only when multiple documents need them.
        Otherwise, define them near the content block and annotate why they are
        local to that document.
\end{itemize}

Always add a brief comment when applying an override; the comment should explain
both the symptom and the fix. Future maintainers can then confirm whether the
original constraint still holds.

\SectionBar{3.5}{Maintenance Checklist}
\label{sec:maintenance-checklist}

Before opening a pull request or tagging a release, walk through this list:
\begin{enumerate}[leftmargin=2em]
  \item \textbf{Clean builds.} Run the textbook and every affected homework
        through \texttt{latexmk}. Confirm that each per-file output directory
        (and \InlineCode{../output/} if you keep using it for the book) only
        contains fresh artefacts.
  \item \textbf{Log audit.} Skim the generated \InlineCode{.log} files for overfull
        boxes, missing references, and font warnings. Appendix sections are most
        prone to margin issues because they reuse the homework typography.
  \item \textbf{Documentation sync.} Keep the prose in
        \InlineCode{projects/textbook/src/chapters/} or the Markdown notes under
        \InlineCode{docs/} up to date so every new macro or workflow is described
        in at least one place.
  \item \textbf{Output hygiene.} Delete stray PDFs, logs, and Synctex files that
        slipped outside the per-file directories. The root
        \InlineCode{.gitignore} already blankets common extensions, but explicit
        cleanup keeps reviewers focused on the meaningful diffs.
  \item \textbf{Cross-project parity.} If you changed a shared module, recompile
        at least one standalone homework to ensure the update behaves identically
        outside the textbook.
\end{enumerate}

Following this routine preserves the repository's goal: every file, directory,
and macro exists for a documented reason, and collaborators can reproduce the
build without guessing at hidden conventions.
